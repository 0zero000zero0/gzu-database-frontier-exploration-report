\section{未来发展趋势}

\subsection{标准化}
随着图数据库的广泛应用, 标准化已经成为推动图数据库进一步发展的关键因素。目前, 图数据库面临着查询语言和数据模型不统一的问题, 这给开发者和企业应用带来了不少困难。因此, 类似于SQL作为关系型数据库的标准查询语言, 图数据库也需要统一的标准, 以简化数据共享、迁移和协作。

\textbf{Cypher 标准化}:Neo4j的Cypher语言目前已经是最流行的图查询语言之一。开放的Cypher查询语言计划(OpenCypher)旨在将Cypher推广为图数据库的通用标准, 使开发者能够更加轻松地在不同平台上实现查询逻辑。

\textbf{Gremlin 与 GQL}:Apache TinkerPop的Gremlin也是一种广泛使用的图查询语言, 侧重于遍历和命令式查询。此外, 国际标准组织(ISO)正在推动图查询语言(GQL)的开发, 旨在为图数据的查询和处理提供一个统一的标准。

目前, 多个厂商(如 Neo4j 和 Amazon Neptune)都开始支持OpenCypher, 推动图数据库查询语言的标准化, 使得图数据库技术的互操作性大幅提升, 企业可以在不同平台之间更加无缝地迁移图数据和应用。

\subsection{多模态}
未来, 图数据库与其他数据模型(如文档、键值对等)的融合将形成多模态数据库, 以满足更加复杂的数据管理需求。多模态数据库不仅能够处理图数据, 还能够结合不同类型的数据, 形成更加全面的数据管理系统。

\textbf{多模态数据融合}:多模态数据库将图数据库与关系型数据库、文档数据库等结合起来, 用于处理多源数据。例如, 将社交网络的图数据与用户行为的文档数据结合, 可以更全面地描绘用户画像, 从而为推荐系统提供更有力的支持\cite{han2021multimodal}。

TigerGraph正在开发集成多模态存储的功能, 将图数据与其他结构化和非结构化数据融合, 以支持企业用户更加复杂的分析需求。


\subsection{4AI}
图数据库与人工智能技术的结合, 特别是图神经网络(Graph Neural Networks, GNN)、检索增强生成(Retrieval-Augmented Generation, RAG)、以及 GraphRAG, 是推动图数据库应用和图数据分析的重要方向。

\textbf{图神经网络(GNN)}
图神经网络是一种处理图结构数据的深度学习模型, 能够学习节点、边及图的特征表示。GNN通过对图中节点的特征进行聚合和传递, 可以有效地捕捉复杂图结构中的信息, 适用于节点分类、链接预测和图分类等任务\cite{wu2020gnn}。Neo4j已经推出了GNN的集成模块, 支持用户直接从图数据库中提取数据进行模型训练。例如, GNN可以用于社交网络中识别关键意见领袖(KOL), 通过对节点的连接关系和属性进行学习, 找出在社交网络中影响力最大的用户。


\textbf{检索增强生成(RAG)}
RAG是一种将知识库与生成式AI结合的技术, 通过实时检索增强生成模型的能力。RAG的核心思想是, 在生成响应之前, 通过检索图数据库中的相关信息, 将其提供给生成模型, 以生成更加准确和知识丰富的回答。例如, GraphRAG结合了图神经网络(GNN)和RAG技术, 能够更智能地处理和生成知识密集型回答。假设在医疗知识图谱中, GraphRAG首先利用GNN对知识图谱进行学习, 提取与患者症状相关的所有实体和关系, 然后结合生成式模型为医生生成诊断建议。这种方法不仅能够通过图结构中的关系找到更有价值的上下文信息, 还能利用生成式模型提供精准且个性化的诊疗方案。这使得GraphRAG在复杂的知识检索和生成任务中, 比传统RAG方法具有更高的准确性和灵活性\cite{jiang2024reasoningenhancedhealthcarepredictionsknowledge}。


