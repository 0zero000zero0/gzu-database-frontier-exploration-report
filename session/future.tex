\section{未来发展趋势}

\subsection{标准化}
随着图数据库的广泛应用,标准化已经成为推动图数据库进一步发展的关键因素。目前,图数据库面临着查询语言和数据模型不统一的问题,这给开发者和企业应用带来了不少困难。因此,类似于 SQL 作为关系型数据库的标准查询语言,图数据库也需要统一的标准,以简化数据共享、迁移和协作。

\textbf{Cypher 标准化}:Neo4j 的 Cypher 语言目前已经是最流行的图查询语言之一。开放的 Cypher 查询语言计划(OpenCypher)旨在将 Cypher 推广为图数据库的通用标准,使开发者能够更加轻松地在不同平台上实现查询逻辑。

\textbf{Gremlin 与 GQL}:Apache TinkerPop 的 Gremlin 也是一种广泛使用的图查询语言,侧重于遍历和命令式查询。此外,国际标准组织(ISO)正在推动**图查询语言(GQL)**的开发,旨在为图数据的查询和处理提供一个统一的标准

目前,多个厂商(如 Neo4j 和 Amazon Neptune)都开始支持 OpenCypher,推动图数据库查询语言的标准化,使得图数据库技术的互操作性大幅提升,企业可以在不同平台之间更加无缝地迁移图数据和应用。

\subsection{多模态}
未来,图数据库与其他数据模型(如文档、键值对等)的融合将形成多模态数据库,以满足更加复杂的数据管理需求。多模态数据库不仅能够处理图数据,还能够结合不同类型的数据,形成更加全面的数据管理系统。

\textbf{多模态数据融合}:多模态数据库将图数据库与关系型数据库、文档数据库等结合起来,用于处理多源数据。例如,将社交网络的图数据与用户行为的文档数据结合,可以更全面地描绘用户画像,从而为推荐系统提供更有力的支持\cite{han2021multimodal}。

TigerGraph 正在开发集成多模态存储的功能,将图数据与其他结构化和非结构化数据融合,以支持企业用户更加复杂的分析需求


\subsection{4AI}
图数据库与人工智能技术的结合,特别是图神经网络(Graph Neural Networks, GNN)、检索增强生成(Retrieval-Augmented Generation, RAG)、以及 GraphRAG,是推动图数据库应用和图数据分析的重要方向。

\textbf{图神经网络(GNN)}
图神经网络是一种处理图结构数据的深度学习模型,能够学习节点、边及图的特征表示。GNN 通过对图中节点的特征进行聚合和传递,可以有效地捕捉复杂图结构中的信息,适用于节点分类、链接预测和图分类等任务\cite{wu2020gnn}。Neo4j 已经推出了 GNN 的集成模块,支持用户直接从图数据库中提取数据进行模型训练。例如,GNN 可以用于社交网络中识别关键意见领袖(KOL),通过对节点的连接关系和属性进行学习,找出在社交网络中影响力最大的用户。


\textbf{检索增强生成(RAG)}
RAG 是一种将知识库与生成式 AI 结合的技术,通过实时检索增强生成模型的能力。RAG 的核心思想是,在生成响应之前,通过检索图数据库中的相关信息,将其提供给生成模型,以生成更加准确和知识丰富的回答。例如,GraphRAG 结合了图神经网络(GNN)和 RAG 技术,能够更智能地处理和生成知识密集型回答。假设在医疗知识图谱中,GraphRAG 首先利用 GNN 对知识图谱进行学习,提取与患者症状相关的所有实体和关系,然后结合生成式模型为医生生成诊断建议。这种方法不仅能够通过图结构中的关系找到更有价值的上下文信息,还能利用生成式模型提供精准且个性化的诊疗方案。这使得 GraphRAG 在复杂的知识检索和生成任务中,比传统 RAG 方法具有更高的准确性和灵活性\cite{jiang2024reasoningenhancedhealthcarepredictionsknowledge}。


